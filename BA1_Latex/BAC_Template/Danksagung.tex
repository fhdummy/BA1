\section*{Danksagung:}

Hier bitte eine entsprechende Danksagung einf�gen (diese Seite mit Danksagung und "`gescheitem Spruch"' kann auf Wunsch auch gel�scht werden). 

An entsprechender Stelle im Dokument sollte aber folgende Textzeile immer eingef�gt werden (am besten als Fu�note): Der vorliegende Text ist auf Basis des \LaTeX{}-Templates aus \cite{gockel07} erstellt. Das Originaltemplate wurde als Basis f�r die Erstellung einer Bachelorarbeit am Studiengang ITS angepasst.

[1] T. \mbox{Gockel}. Form der wissenschaftlichen Ausarbeitung. Springer-Verlag, Heidelberg, 2008. Begleitende Materialien unter \url{http://www.formbuch.de}.

\textbf{Bemerkung zur Lizenz:} Das Template darf angepasst, ver�ndert, erweitert und auch kommerziell vertrieben werden. Die einzige Aufllage ist, dass die Quelle des Templates in den Literaturquellen genannt und im Text als Quelle referenziert wird. Hierzu ist dem Text ein kurzer Satz (wie oben angef�hrt) beizuf�gen und am Ende ist eine Quelle einzuf�gen. 

Die Erweiterungen und Anpassungen am Template als Basis f�r eine ITS Bachelorarbeit wurden von Karl Entacher und Simon Kranzer durchgef�hrt.


\vspace*{\fill}

\begin{center}
\begin{quote}
Warum die Ingenieure auf der niedrigsten Ebene der Entscheidungsprozesse eingestuft werden, wei� ich nicht, aber dies scheint ein allgemeines Gesetz zu sein: Jene, die etwas �ber die wirkliche Welt wissen, bilden in diesen gro�en Organisationen die unterste Stufe, und jene, die nur wissen, wie man andere Leute beeinflussen kann, indem man ihnen sagt, wie sch�n die Welt im Idealfall sein k�nnte, sind an der Spitze.
\end{quote}
Richard P. Feynman
\end{center}

\newpage