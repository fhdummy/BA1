\chapter{Zusammenfassung und Ausblick}
%%%%%%%%%%%%%%%%%%%%%%%%%%%%%%%%%%%%%%
Die "`Conclusio"' dient der Abrundung der wissenschaftlichen Bachelorarbeit. Sie umfasst in komprimierter Form die wesentlichen Aussagen zur L�sung der Aufgabe bzw. die knappe Darstellung von erarbeiteten Thesen. Der/die VerfasserIn kann hier deutlich machen, dass das in der Einleitung angek�ndigte Anliegen der Arbeit erreicht worden ist. 

Weiters gibt das abschlie�ende Kapitel Raum f�r kritische Anmerkungen und kann dar�ber hinaus dazu genutzt werden, den LeserInnen Informationen �ber zu erwartende Entwicklungen auf dem behandelten Themengebiet zu liefern.

\section{Literaturangaben}
Verweise auf Literatur werden in einem separaten Abschnitt am Ende der Bachelorarbeit, beginnend auf einer neuen Seite, mit dem Titel "`Literaturverzeichnis"' angef�hrt. Literaturquellen sind vollst�ndig zu benennen und haben je nach Art der Quelle einen bestimmten Stil. Die Literatur wird im File \texttt{bibliographie.bib} verwaltet. Die BibTeX-Eintr�ge k�nnen mit Hilfe eines Standardeditors editiert werden. Alternativ dazu kann man ein Literaturverwaltungsprogramm wie JabRef \cite{jabref11} verwenden. F�r Fragen betreffend Quellenangaben zu B�chern, Artikel, Konferenzberichten, Normen, RFCs, White-Papers, Poster, Internetquellen usw. siehe \cite{Ent10}. F�r weiterf�hrende Literatur zu \LaTeX{} siehe z.B.~\cite{braune09,kopka05,lamprecht00}.
























%